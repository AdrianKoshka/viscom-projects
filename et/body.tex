\section{Introduction}
\paragraph{}
Typefaces are a crucial part of everyday life. While most people don't even think
about typefaces, they can make or break something. Whether it be a piece of 
advertisement, the user interface (UI) of a program, or a simple flyer for a
neighborhood event, typefaces can have a great effect on how something is
perceived.
\par

\subsection{What exactly is a typeface?}
\paragraph{}
In the previous paragraph, you probably noticed the use of the word ``typefaces''
instead of the word ``fonts''. You're probably wondering what the difference is
between a ``typeface'', and a ``font'' is. I stumbled upon this explanation of
``typeface'' vs. ``font''.

\begin{displayquote}
	A typeface is a family of fonts (very often by the same designer). within a
	typeface there will be fonts of varying weights or other variants. E.g.,
	light, bold, semi-bold, condensed, italic, etc. Each such variation is a
	different font. The only evolution is terminology that results from the
	transition from metal-cast to digital fonts is that (point) size is no
	longer fixed. \cite{jontan2008}
\end{displayquote}

To boil it down further, a typeface is a family of fonts, each font within the
typeface having a unique characteristic to it.
\par

\subsection{What is this paper about?}
\paragraph{}
This paper aims to serve as a brief overview as to why some typefaces have
become commonplace, what makes a ``good'' or ``bad'', and why some have been
exiled into ridicule.
\par

\newpage

\subsection{Terminology}
\paragraph{}
	\begin{description}
		\item [Font:] A complete assortment of type of one style. \cite{Dict2017font}
		\item [Font families:] Collections of closely related typeface designs. \cite{wiki:typeface2017}
			\begin{itemize}
				\item Example: the OpenDyslexic font family has \textbf{bold} and \textit{italic} fonts within it.
			\end{itemize}
		\item [Glyph:] A glyph is an element of writing.  \cite{wiki:glyph2017}
			\begin{itemize}
				\item Example: the ligature 'æ' in archæology represents a and e letters. \cite{wiki:glyph2017}
			\end{itemize}
		\item [Serif:] A decorative line used to embilesh a letter, and make it easier to read.
	\end{description}
\par


%\section{A brief history on the Evolution of Typefaces.}
%\paragraph{}
%Type has quite a long history, so for the purposes of this paper, I'll try to just
%cover some of the main points in type history that pertain to this paper.
%\par

\section{Typical Typefaces}
\subsection{Helvetica}
%\begin{figure}[h]
%	\centering
%	\includegraphics[scale=0.25]{images/HelveticaSpecimenCH.pdf}
%	\caption{\cite{wiki:helvetica-image}}
%\end{figure}
\subsubsection{A Brief History}
\paragraph{}

\par

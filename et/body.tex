\section{Introduction}
\paragraph{Typefaces}
are a crucial part of everyday life. While most people don't even think about
typefaces, they can make or break something. Whether it be a piece of advertisement,
the user interface (UI) of a program, or a simlpe flyer for a neighbourhood
event, typefaces can have a great effect on how something is perceived.
\par

\subsection{What exactly is a typeface?}
\paragraph{}
In the previous paragraph, you probably noticed the use of the word ``typefaces''
instead of the word ``fonts''. You're probably wondering what the difference is
between a ``typeface'', and a ``font'' is. I stumbled upon this explanation of
``typeface'' vs. ``font''.

\begin{displayquote}
	A typeface is a family of fonts (very often by the same designer). within a
	typeface there will be fonts of varying weights or other varients. E.g.,
	light, bold, semi-bold, condensed, italic, etc. Each such variation is a
	differen font. The only evolution is terminology that results from the
	transition from metal-cast to digital fonts is that (point) size is no
	longer fixed. \cite{jontan2008}
\end{displayquote}

To boil it down further, a typeface is a family of fonts, each font within the
typeface having a unique characteristic to it.

\par
